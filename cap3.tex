%\cleardoublepage
\pagestyle{fancy}

\chapter{Estado da arte}
\label{cap3}

Use comandos predifinidos para evitar repetir siglas longas e complicadas. Um
exemplo é \hws{}. Apresente símbolos e formulações matemáticas no meio do texto
como, por exemplo $v \in R$. Mostre uma equação importante numa linha a parte
como:

\begin{equation}
	f_w = \sum_{i \in w} v_i (\sum_{j \in w_i} p_{ij}).
\end{equation}

Note o ponto final na equação acima. Você pode escrever texto em \textit{itálico
}) também. Mais uma equação complicadinha a seguir.
\begin{equation}
	h(x, f, p, \theta) =
	\left\{
		\begin{array}{l l}
			1 & \text{se } pf(x) < p\theta\\
			-1 & \text{caso contrário}\\
		\end{array}
	\right. .
\end{equation}

É isso\ldots