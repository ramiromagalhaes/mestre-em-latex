%\cleardoublepage
\pagestyle{fancy}

\chapter{Seu tema}
\label{cap4}

Apresento a seguir uma bela figura. Ela está armazenada num diretório de imagens
que configurei no arquivo ``configuracoes.tex''. Note o uso de legenda.

\begin{figure}[htbp]
	\centering
	\includegraphics[width=\textwidth]{ilustracao-shen}
	\caption[Uma imagem esquisita \cite{opencvsite}.]{Uma imagem esquisita
	\cite{opencvsite}.}
	\label{fig:shen}
\end{figure}

Referencie sua figura \ref{fig:shen}. Inicie, a seguir, uma nova seção.

\section{Discussão}
\label{cap4:discussao}

Mais conversa e podemos ter outra figura.

\begin{figure}[htbp]
	\centering
	\includegraphics[width=\textwidth]{ilustracao-pavani}
	\caption[Outra imagem estranha.]{Posso colocar um texto diferente. Posso
	usar letras gregas (ex.:  $\mu$) em vários lugares.}
	\label{fig:pavani}
\end{figure}

E, para mudar a narrativa, crie uma lista de tópicos\ldots

\begin{enumerate}
	\item Primeiro tópico
	\item Outro tópico
\end{enumerate}

Encerre.
