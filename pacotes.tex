%%% Pacotes utilizados %%%

%% Codificação e formatação básica do LaTeX
% Suporte para português (hifenação e caracteres especiais)
\usepackage[brazilian,english]{babel}

% Codificação do arquivo
\usepackage[utf8]{inputenc}

% Mapear caracteres especiais no PDF
\usepackage{cmap}

% Codificação da fonte
\usepackage[T1]{fontenc}

% Essencial para colocar funções e outros símbolos matemáticos
\usepackage{amsmath,amssymb,amsfonts,textcomp}

\usepackage[portuguese]{algorithm2e}

%% Layout
% Customização do layout da página, margens espelhadas
\usepackage[twoside]{geometry}

%\usepackage{layout}

% Para definir espaçamento entre as linhas
\usepackage{setspace}

%% Elementos Gráficos
% Para incluir figuras (pacote extendido)
\usepackage[]{graphicx}

% Suporte a cores
\usepackage{color}

% Criar figura dividida em subfiguras
\usepackage{subfig}

% Customizar as legendas de figuras e tabelas
\usepackage{caption}

% Criar ambientes com 2 ou mais colunas
\usepackage{multicol}

% Ative o comando abaixo se quiser colocar figuras de fundo (e.g., capa)
%\usepackage{wallpaper}

%% Tabelas
% Elementos extras para formatação de tabelas
\usepackage{array}

% Tabelas com qualidade de publicação
\usepackage{booktabs}

% Para criar tabelas maiores que uma página
\usepackage{longtable}

% adicionar tabelas e figuras como landscape
\usepackage{lscape}

%% Lista de Abreviações
% Cria lista de abreviações
\usepackage[notintoc,portuguese]{nomencl}

%% Notas de rodapé
% Lidar com notas de rodapé em diversas situações
\usepackage{footnote}

% Conta o número de páginas
\usepackage{lastpage}

% Adicionar bibliografia, índice e conteúdo na Tabela de conteúdo
% Não inclui lista de tabelas e figuras no índice
\usepackage[nottoc,notlof,notlot]{tocbibind}

%% Pontuação e unidades
% Posicionar inteligentemente a vírgula como separador decimal
\usepackage{icomma}

% Formatar as unidades com as distâncias corretas
\usepackage[tight]{units}

%% Cabeçalho e rodapé
% Controlar os cabeçalhos e rodapés
\usepackage{fancyhdr}

%% Links dinâmicos
% Suporte para hipertexto, links para referências e figuras
\usepackage{hyperref}
